%=============================================================================
% Chapter: Release Process
%=============================================================================

\chapter{Release Process}
\label{Release Process}

The release process for \parflow{} follows the a standard GitHub
release process.  The following steps should be followed for creating a new
\parflow{} release.

\begin{enumerate}
	\item Create branch/fork for generating the release
	\item Edit the RELEASE-NOTES.md file
	\item Regenerate the \parflow{} User Manual
        \item Commit release file changes
        \item Create a pull request for the branch/fork
        \item Merge the pull request
        \item Generate a release on GitHub
\end{enumerate}


\section{Create branch}

Use standard Git/GitHub commands to create a branch for editing some files for the release.

\section{Edit files}

Edit the RELEASE-NOTES.md file to add notes about what was changed in
this release.  Old release notes should be deleted, we do not keep a
running account of release information.

Edit VERSION file with current version.

Edit \code{./docs/manuals/user_manual.tex} and insert the current date
into the subtitle.

\section{Regenerate the \parflow{} User Manual}

The repository contains a PDF of the User Manual.  This should be
regenerated for the release so it is up-to-date.  This step requires a
Latex install to perform.  By default the configure system will use
the current tag and git hash to label the release.  By moving the .git
directory the cmake configure will use the version specified in the
just edited VERSION file.

\begin{display}\begin{verbatim}
  cd <directory with ParFlow source>
  mv .git git
  mkdir build
  cd build
  cmake -DPARFLOW_ENABLE_LATEX=TRUE ../parflow/
  make
  cp ./docs/manuals/user_manual.pdf ../parflow/
  cd ..
  mv git .git
\end{verbatim}\end{display}
  
\section{Commit release file changes}

Use standard git add and git commit commands to add the modified files
to the release branch/fork.

\section{Create a pull request for the branch/fork}

Use GitHub to create a pull request for the release branch.
  
\section{Merge the pull request}

Use GitHub to create a pull request for the release branch.
  
\section{Generate a release on GitHub}

On the GitHub \code{https://github.com/parflow/parflow/releases} page use
``Draft a new release'' to create the release.

Version tag should have format of ``vX.Z.Z'' version.  Release title
should have format of ``ParFlow Version X.Y.Z''.  The GitHub release
description can be copied from the release notes markdown file that
was created in a proir step.
